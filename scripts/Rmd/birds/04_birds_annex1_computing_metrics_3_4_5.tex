% Options for packages loaded elsewhere
\PassOptionsToPackage{unicode}{hyperref}
\PassOptionsToPackage{hyphens}{url}
%
\documentclass[
]{article}
\usepackage{amsmath,amssymb}
\usepackage{iftex}
\ifPDFTeX
  \usepackage[T1]{fontenc}
  \usepackage[utf8]{inputenc}
  \usepackage{textcomp} % provide euro and other symbols
\else % if luatex or xetex
  \usepackage{unicode-math} % this also loads fontspec
  \defaultfontfeatures{Scale=MatchLowercase}
  \defaultfontfeatures[\rmfamily]{Ligatures=TeX,Scale=1}
\fi
\usepackage{lmodern}
\ifPDFTeX\else
  % xetex/luatex font selection
\fi
% Use upquote if available, for straight quotes in verbatim environments
\IfFileExists{upquote.sty}{\usepackage{upquote}}{}
\IfFileExists{microtype.sty}{% use microtype if available
  \usepackage[]{microtype}
  \UseMicrotypeSet[protrusion]{basicmath} % disable protrusion for tt fonts
}{}
\makeatletter
\@ifundefined{KOMAClassName}{% if non-KOMA class
  \IfFileExists{parskip.sty}{%
    \usepackage{parskip}
  }{% else
    \setlength{\parindent}{0pt}
    \setlength{\parskip}{6pt plus 2pt minus 1pt}}
}{% if KOMA class
  \KOMAoptions{parskip=half}}
\makeatother
\usepackage{xcolor}
\usepackage[margin=1in]{geometry}
\usepackage{color}
\usepackage{fancyvrb}
\newcommand{\VerbBar}{|}
\newcommand{\VERB}{\Verb[commandchars=\\\{\}]}
\DefineVerbatimEnvironment{Highlighting}{Verbatim}{commandchars=\\\{\}}
% Add ',fontsize=\small' for more characters per line
\usepackage{framed}
\definecolor{shadecolor}{RGB}{248,248,248}
\newenvironment{Shaded}{\begin{snugshade}}{\end{snugshade}}
\newcommand{\AlertTok}[1]{\textcolor[rgb]{0.94,0.16,0.16}{#1}}
\newcommand{\AnnotationTok}[1]{\textcolor[rgb]{0.56,0.35,0.01}{\textbf{\textit{#1}}}}
\newcommand{\AttributeTok}[1]{\textcolor[rgb]{0.13,0.29,0.53}{#1}}
\newcommand{\BaseNTok}[1]{\textcolor[rgb]{0.00,0.00,0.81}{#1}}
\newcommand{\BuiltInTok}[1]{#1}
\newcommand{\CharTok}[1]{\textcolor[rgb]{0.31,0.60,0.02}{#1}}
\newcommand{\CommentTok}[1]{\textcolor[rgb]{0.56,0.35,0.01}{\textit{#1}}}
\newcommand{\CommentVarTok}[1]{\textcolor[rgb]{0.56,0.35,0.01}{\textbf{\textit{#1}}}}
\newcommand{\ConstantTok}[1]{\textcolor[rgb]{0.56,0.35,0.01}{#1}}
\newcommand{\ControlFlowTok}[1]{\textcolor[rgb]{0.13,0.29,0.53}{\textbf{#1}}}
\newcommand{\DataTypeTok}[1]{\textcolor[rgb]{0.13,0.29,0.53}{#1}}
\newcommand{\DecValTok}[1]{\textcolor[rgb]{0.00,0.00,0.81}{#1}}
\newcommand{\DocumentationTok}[1]{\textcolor[rgb]{0.56,0.35,0.01}{\textbf{\textit{#1}}}}
\newcommand{\ErrorTok}[1]{\textcolor[rgb]{0.64,0.00,0.00}{\textbf{#1}}}
\newcommand{\ExtensionTok}[1]{#1}
\newcommand{\FloatTok}[1]{\textcolor[rgb]{0.00,0.00,0.81}{#1}}
\newcommand{\FunctionTok}[1]{\textcolor[rgb]{0.13,0.29,0.53}{\textbf{#1}}}
\newcommand{\ImportTok}[1]{#1}
\newcommand{\InformationTok}[1]{\textcolor[rgb]{0.56,0.35,0.01}{\textbf{\textit{#1}}}}
\newcommand{\KeywordTok}[1]{\textcolor[rgb]{0.13,0.29,0.53}{\textbf{#1}}}
\newcommand{\NormalTok}[1]{#1}
\newcommand{\OperatorTok}[1]{\textcolor[rgb]{0.81,0.36,0.00}{\textbf{#1}}}
\newcommand{\OtherTok}[1]{\textcolor[rgb]{0.56,0.35,0.01}{#1}}
\newcommand{\PreprocessorTok}[1]{\textcolor[rgb]{0.56,0.35,0.01}{\textit{#1}}}
\newcommand{\RegionMarkerTok}[1]{#1}
\newcommand{\SpecialCharTok}[1]{\textcolor[rgb]{0.81,0.36,0.00}{\textbf{#1}}}
\newcommand{\SpecialStringTok}[1]{\textcolor[rgb]{0.31,0.60,0.02}{#1}}
\newcommand{\StringTok}[1]{\textcolor[rgb]{0.31,0.60,0.02}{#1}}
\newcommand{\VariableTok}[1]{\textcolor[rgb]{0.00,0.00,0.00}{#1}}
\newcommand{\VerbatimStringTok}[1]{\textcolor[rgb]{0.31,0.60,0.02}{#1}}
\newcommand{\WarningTok}[1]{\textcolor[rgb]{0.56,0.35,0.01}{\textbf{\textit{#1}}}}
\usepackage{graphicx}
\makeatletter
\def\maxwidth{\ifdim\Gin@nat@width>\linewidth\linewidth\else\Gin@nat@width\fi}
\def\maxheight{\ifdim\Gin@nat@height>\textheight\textheight\else\Gin@nat@height\fi}
\makeatother
% Scale images if necessary, so that they will not overflow the page
% margins by default, and it is still possible to overwrite the defaults
% using explicit options in \includegraphics[width, height, ...]{}
\setkeys{Gin}{width=\maxwidth,height=\maxheight,keepaspectratio}
% Set default figure placement to htbp
\makeatletter
\def\fps@figure{htbp}
\makeatother
\setlength{\emergencystretch}{3em} % prevent overfull lines
\providecommand{\tightlist}{%
  \setlength{\itemsep}{0pt}\setlength{\parskip}{0pt}}
\setcounter{secnumdepth}{-\maxdimen} % remove section numbering
\ifLuaTeX
  \usepackage{selnolig}  % disable illegal ligatures
\fi
\usepackage{bookmark}
\IfFileExists{xurl.sty}{\usepackage{xurl}}{} % add URL line breaks if available
\urlstyle{same}
\hypersetup{
  pdftitle={ Data mobilisation from GBIF  to the EBV Data Portal},
  hidelinks,
  pdfcreator={LaTeX via pandoc}}

\title{ Data mobilisation from GBIF to the EBV Data Portal}
\usepackage{etoolbox}
\makeatletter
\providecommand{\subtitle}[1]{% add subtitle to \maketitle
  \apptocmd{\@title}{\par {\large #1 \par}}{}{}
}
\makeatother
\subtitle{ Calculation of Metrics for the Birds Directive Annex I Using
Occurrence Cubes (Part II)}
\author{true}
\date{2024-08-22}

\begin{document}
\maketitle

\subsection{}\label{section}

\subsubsection{Introduction}\label{introduction}

In this notebook, we calculate simple metrics for the EU Annex I Birds
Directive based on species occurrence in GBIF. For this, a bird
occurrence cube (see Notebook 01 in this repository) was previously
created using the
\href{https://techdocs.gbif.org/en/data-use/data-cubes}{occurrence cube
software} developed by GBIF under the
\href{https://b-cubed.eu/}{Biodiversity Building Blocks for Policy} (B3)
Project. Details of the data query in GBIF are available at
\href{https://doi.org/10.15468/dl.uh84tp/}{doi: 10.15468/dl.uh84tp}. In
this notebook we calculate metrics 4, 5 and 6 listed below:

\begin{itemize}
\tightlist
\item
  Month with the highest total number of occurrences across all years
\item
  Month with the second-highest total number of occurrences across all
  years
\item
  Month with the third-highest total number of occurrences across all
  years
\end{itemize}

\emph{Note: This series of notebooks is part of the results of Task 3.3
of the \href{https://b-cubed.eu/}{Biodiversity Building Blocks for
Policy} project funded by the European Union's Horizon Europe Research
and Innovation Programme (ID No 101059592). Additional notebooks
exploring the results and calculating simple metrics are also available
in the same repository.}

\subsubsection{Load Library and Input
Data}\label{load-library-and-input-data}

We start by loading all the libraries needed in this notebook.

\begin{Shaded}
\begin{Highlighting}[]
\FunctionTok{rm}\NormalTok{(}\AttributeTok{list=}\FunctionTok{ls}\NormalTok{())}
\FunctionTok{gc}\NormalTok{()}
\end{Highlighting}
\end{Shaded}

\begin{verbatim}
##          used (Mb) gc trigger (Mb) max used (Mb)
## Ncells 470297 25.2    1021605 54.6   644200 34.5
## Vcells 846515  6.5    8388608 64.0  1636226 12.5
\end{verbatim}

\begin{Shaded}
\begin{Highlighting}[]
\CommentTok{\# Load requiered libraries}
\FunctionTok{library}\NormalTok{(b3gbi) }\CommentTok{\# for csv occurrence cubes}
\FunctionTok{library}\NormalTok{(purrr) }\CommentTok{\# for data summary and grouping}
\FunctionTok{library}\NormalTok{(here)}
\FunctionTok{library}\NormalTok{(dplyr)}
\FunctionTok{library}\NormalTok{(lubridate) }\CommentTok{\# for dates}
\FunctionTok{library}\NormalTok{(terra) }\CommentTok{\# for raster}
\FunctionTok{library}\NormalTok{(ncdf4)}
\FunctionTok{library}\NormalTok{(ggplot2)}
\FunctionTok{library}\NormalTok{(sf)}
\FunctionTok{library}\NormalTok{(stringr)}
\FunctionTok{library}\NormalTok{(viridis)}
\end{Highlighting}
\end{Shaded}

Now we load all input data obtained in the previous notebooks. These
are:

\begin{itemize}
\tightlist
\item
  The occurrence cube of the Birds Directive Annex I obtained previously
  through GBIF API.
\item
  Taxonomic information of all analysed species.
\item
  Inputs related to the EEA grid at 10 km for rasterisation.
\end{itemize}

\begin{Shaded}
\begin{Highlighting}[]
\NormalTok{occcube }\OtherTok{\textless{}{-}} \StringTok{"0062979{-}240626123714530"}

\CommentTok{\# Load occurrence cube using b3gbi}
\NormalTok{cin }\OtherTok{\textless{}{-}} \FunctionTok{process\_cube}\NormalTok{(}\FunctionTok{here}\NormalTok{(}\FunctionTok{paste0}\NormalTok{(}\StringTok{"output/datacubes/csv/birds/"}\NormalTok{, occcube,}\StringTok{".csv"}\NormalTok{)))}

\CommentTok{\# Import the taxonomy of all species listed in the Birds Directive Annex 1 based on gbif backbone taxonomy}
\NormalTok{gbif\_tax }\OtherTok{\textless{}{-}} \FunctionTok{read.csv}\NormalTok{(}\FunctionTok{here}\NormalTok{(}\StringTok{"input/data/birds/taxonomy/list193birds\_directive\_annexi\_allaccepted\_usagekey\_gbif.csv"}\NormalTok{))}

\CommentTok{\# Load reference EEA grid in raster format}
\NormalTok{gridin }\OtherTok{\textless{}{-}} \FunctionTok{rast}\NormalTok{(}\FunctionTok{here}\NormalTok{(}\StringTok{"input/grid/eeagrid\_10K.tif"}\NormalTok{))}
\NormalTok{res }\OtherTok{\textless{}{-}} \FunctionTok{res}\NormalTok{(gridin)[}\DecValTok{1}\NormalTok{] }\CommentTok{\#resolution of reference raster}

\CommentTok{\# Load precomputed centroids for EEA 10 km grid}
\NormalTok{coorin }\OtherTok{\textless{}{-}} \FunctionTok{read.csv}\NormalTok{(}\FunctionTok{here}\NormalTok{(}\StringTok{"input/grid/centroids/eeagrid\_centroids\_10K.csv"}\NormalTok{))}

\CommentTok{\# Replace by corresponding column name of input dataset}
\FunctionTok{colnames}\NormalTok{(coorin)[}\FunctionTok{colnames}\NormalTok{(coorin) }\SpecialCharTok{==} \StringTok{"eeacellcode"}\NormalTok{] }\OtherTok{\textless{}{-}} \StringTok{"cellCode"}

\CommentTok{\# Import and convert the EU borders to a data frame}
\NormalTok{borders\_eu }\OtherTok{\textless{}{-}} \FunctionTok{st\_read}\NormalTok{(}\FunctionTok{here}\NormalTok{(}\StringTok{"input/grid/shp/NUTS2021\_3035.shp"}\NormalTok{))}
\end{Highlighting}
\end{Shaded}

\begin{verbatim}
## Reading layer `NUTS2021_3035' from data source 
##   `H:\B3\B-Cubed_data_mobilization\input\grid\shp\NUTS2021_3035.shp' 
##   using driver `ESRI Shapefile'
## Simple feature collection with 44 features and 27 fields
## Geometry type: MULTIPOLYGON
## Dimension:     XY
## Bounding box:  xmin: 943758.8 ymin: 941658.1 xmax: 7316569 ymax: 6405005
## Projected CRS: ETRS89-extended / LAEA Europe
\end{verbatim}

We now identify which species of the Birds Directive Annex I have data
in our species occurrence cubes.

\begin{Shaded}
\begin{Highlighting}[]
\CommentTok{\# Find number of species}
\NormalTok{spskey }\OtherTok{\textless{}{-}} \FunctionTok{unique}\NormalTok{(cin[[}\StringTok{"data"}\NormalTok{]][[}\StringTok{"taxonKey"}\NormalTok{]]) }\CommentTok{\# "taxonKey" is the name of the species key in b3bgi}
\end{Highlighting}
\end{Shaded}

For metrics 1 to 3, see the accompanying notebook in the same
repository.

\paragraph{\texorpdfstring{\textbf{Metrics 4, 5 and 6:} Months with the
highest, second-highest, and third-highest total number of GBIF
occurrences across all
years}{Metrics 4, 5 and 6: Months with the highest, second-highest, and third-highest total number of GBIF occurrences across all years}}\label{metrics-4-5-and-6-months-with-the-highest-second-highest-and-third-highest-total-number-of-gbif-occurrences-across-all-years}

We identify the months with the highest (Metric 4), second highest
(Metric 5), and third highest (Metric 6) number of records occurrences
at GBIF.

\begin{Shaded}
\begin{Highlighting}[]
\NormalTok{maxrec\_month123 }\OtherTok{\textless{}{-}} \ControlFlowTok{function}\NormalTok{(cin, gridin, coorin)\{}
  \CommentTok{\# Find number of species}
\NormalTok{  spskey }\OtherTok{\textless{}{-}} \FunctionTok{unique}\NormalTok{(cin[[}\StringTok{"data"}\NormalTok{]][[}\StringTok{"taxonKey"}\NormalTok{]])}

  \CommentTok{\# Create empty raster for metric 4}
\NormalTok{  r4 }\OtherTok{\textless{}{-}} \FunctionTok{rast}\NormalTok{(}\FunctionTok{ext}\NormalTok{(gridin), }\AttributeTok{resolution=}\FunctionTok{res}\NormalTok{(gridin), }\AttributeTok{nlyrs=}\FunctionTok{length}\NormalTok{(spskey), }\AttributeTok{crs=}\FunctionTok{crs}\NormalTok{(gridin))}
  \FunctionTok{values}\NormalTok{(r4) }\OtherTok{\textless{}{-}} \ConstantTok{NA}
  
  \CommentTok{\# Create empty raster for metric 5}
\NormalTok{  r5 }\OtherTok{\textless{}{-}} \FunctionTok{rast}\NormalTok{(}\FunctionTok{ext}\NormalTok{(gridin), }\AttributeTok{resolution=}\FunctionTok{res}\NormalTok{(gridin), }\AttributeTok{nlyrs=}\FunctionTok{length}\NormalTok{(spskey), }\AttributeTok{crs=}\FunctionTok{crs}\NormalTok{(gridin))}
  \FunctionTok{values}\NormalTok{(r5) }\OtherTok{\textless{}{-}} \ConstantTok{NA}
  
  \CommentTok{\# Create empty raster for metric 6}
\NormalTok{  r6 }\OtherTok{\textless{}{-}} \FunctionTok{rast}\NormalTok{(}\FunctionTok{ext}\NormalTok{(gridin), }\AttributeTok{resolution=}\FunctionTok{res}\NormalTok{(gridin), }\AttributeTok{nlyrs=}\FunctionTok{length}\NormalTok{(spskey), }\AttributeTok{crs=}\FunctionTok{crs}\NormalTok{(gridin))}
  \FunctionTok{values}\NormalTok{(r6) }\OtherTok{\textless{}{-}} \ConstantTok{NA}

  \CommentTok{\# i \textless{}{-} 24}

  \ControlFlowTok{for}\NormalTok{ (i }\ControlFlowTok{in} \DecValTok{56}\SpecialCharTok{:}\DecValTok{58}\NormalTok{)\{ }\CommentTok{\# 1:length(spskey))\{}
  \CommentTok{\# print(i)}

  \CommentTok{\# Subset one species}
\NormalTok{  spsi }\OtherTok{\textless{}{-}}\NormalTok{ cin[[}\StringTok{"data"}\NormalTok{]][cin[[}\StringTok{"data"}\NormalTok{]]}\SpecialCharTok{$}\NormalTok{taxonKey }\SpecialCharTok{==}\NormalTok{ spskey[i], ]}

  \CommentTok{\# Convert date from character to numeric}
\NormalTok{  todates }\OtherTok{\textless{}{-}} \FunctionTok{as.data.frame}\NormalTok{(}\FunctionTok{str\_split}\NormalTok{(spsi}\SpecialCharTok{$}\NormalTok{yearMonth, }\StringTok{"{-}"}\NormalTok{, }\AttributeTok{simplify =} \ConstantTok{TRUE}\NormalTok{))}
  \FunctionTok{colnames}\NormalTok{(todates) }\OtherTok{\textless{}{-}} \FunctionTok{c}\NormalTok{(}\StringTok{"year"}\NormalTok{, }\StringTok{"month"}\NormalTok{)}

  \CommentTok{\# Add year and month columns separate to the initial data}
\NormalTok{  spsi}\SpecialCharTok{$}\NormalTok{year }\OtherTok{\textless{}{-}}\NormalTok{ todates}\SpecialCharTok{$}\NormalTok{year}
\NormalTok{  spsi}\SpecialCharTok{$}\NormalTok{month }\OtherTok{\textless{}{-}}\NormalTok{ todates}\SpecialCharTok{$}\NormalTok{month}

  \CommentTok{\# Aggregate occurrences by cell and month}
\NormalTok{  month\_occ }\OtherTok{\textless{}{-}}\NormalTok{ spsi }\SpecialCharTok{\%\textgreater{}\%}
      \FunctionTok{group\_by}\NormalTok{(cellCode, month) }\SpecialCharTok{\%\textgreater{}\%}
      \FunctionTok{summarize}\NormalTok{(}\AttributeTok{total\_occurrences =} \FunctionTok{sum}\NormalTok{(obs), }\AttributeTok{cellCode =} \FunctionTok{first}\NormalTok{(cellCode)) }\SpecialCharTok{\%\textgreater{}\%}
      \FunctionTok{ungroup}\NormalTok{()}

    \CommentTok{\# Calculate metrics 4, 5, and 6 (1st, 2nd, and 3rd highest months)}
\NormalTok{    metric4 }\OtherTok{\textless{}{-}}\NormalTok{ month\_occ }\SpecialCharTok{\%\textgreater{}\%}
      \FunctionTok{group\_by}\NormalTok{(cellCode) }\SpecialCharTok{\%\textgreater{}\%}
      \FunctionTok{slice\_max}\NormalTok{(total\_occurrences, }\AttributeTok{n =} \DecValTok{1}\NormalTok{, }\AttributeTok{with\_ties =} \ConstantTok{FALSE}\NormalTok{) }\SpecialCharTok{\%\textgreater{}\%}
      \FunctionTok{ungroup}\NormalTok{()}

\NormalTok{    metric5 }\OtherTok{\textless{}{-}}\NormalTok{ month\_occ }\SpecialCharTok{\%\textgreater{}\%}
      \FunctionTok{group\_by}\NormalTok{(cellCode) }\SpecialCharTok{\%\textgreater{}\%}
      \FunctionTok{slice\_max}\NormalTok{(total\_occurrences, }\AttributeTok{n =} \DecValTok{2}\NormalTok{, }\AttributeTok{with\_ties =} \ConstantTok{FALSE}\NormalTok{) }\SpecialCharTok{\%\textgreater{}\%}
      \FunctionTok{slice\_min}\NormalTok{(total\_occurrences, }\AttributeTok{n =} \DecValTok{1}\NormalTok{, }\AttributeTok{with\_ties =} \ConstantTok{FALSE}\NormalTok{) }\SpecialCharTok{\%\textgreater{}\%}
      \FunctionTok{ungroup}\NormalTok{()}

\NormalTok{    metric6 }\OtherTok{\textless{}{-}}\NormalTok{ month\_occ }\SpecialCharTok{\%\textgreater{}\%}
      \FunctionTok{group\_by}\NormalTok{(cellCode) }\SpecialCharTok{\%\textgreater{}\%}
      \FunctionTok{slice\_max}\NormalTok{(total\_occurrences, }\AttributeTok{n =} \DecValTok{3}\NormalTok{, }\AttributeTok{with\_ties =} \ConstantTok{FALSE}\NormalTok{) }\SpecialCharTok{\%\textgreater{}\%}
      \FunctionTok{slice\_min}\NormalTok{(total\_occurrences, }\AttributeTok{n =} \DecValTok{1}\NormalTok{, }\AttributeTok{with\_ties =} \ConstantTok{FALSE}\NormalTok{) }\SpecialCharTok{\%\textgreater{}\%}
      \FunctionTok{ungroup}\NormalTok{()}

\NormalTok{  metricx\_coor }\OtherTok{\textless{}{-}} \ControlFlowTok{function}\NormalTok{(metricx, coorin, res)\{}
    
    \CommentTok{\# Merge pixel coordinates with the corresponing EEA grid ID pre{-}rasterisation}
\NormalTok{    metriccoorx }\OtherTok{\textless{}{-}} \FunctionTok{merge}\NormalTok{(metricx[,}\FunctionTok{c}\NormalTok{(}\StringTok{"month"}\NormalTok{, }\StringTok{"cellCode"}\NormalTok{)], coorin, }\AttributeTok{by=}\StringTok{"cellCode"}\NormalTok{)}
    \CommentTok{\# print(length(unique(metriccoorx$x)))}

    \ControlFlowTok{if}\NormalTok{(}\FunctionTok{length}\NormalTok{(}\FunctionTok{unique}\NormalTok{(metriccoorx}\SpecialCharTok{$}\NormalTok{x)) }\SpecialCharTok{\textless{}} \DecValTok{10}\NormalTok{)\{}
      \CommentTok{\#add second empty point}
\NormalTok{      x }\OtherTok{\textless{}{-}}\NormalTok{ metriccoorx}\SpecialCharTok{$}\NormalTok{x[}\DecValTok{1}\NormalTok{] }\SpecialCharTok{+}\NormalTok{ res}
\NormalTok{      y }\OtherTok{\textless{}{-}}\NormalTok{ metriccoorx}\SpecialCharTok{$}\NormalTok{y[}\DecValTok{1}\NormalTok{] }\SpecialCharTok{+}\NormalTok{ res}
\NormalTok{      metriccoorx }\OtherTok{\textless{}{-}} \FunctionTok{rbind}\NormalTok{(metriccoorx, }\FunctionTok{c}\NormalTok{(}\ConstantTok{NA}\NormalTok{, }\ConstantTok{NA}\NormalTok{ ,}\ConstantTok{NA}\NormalTok{, x, y))}
\NormalTok{    \}}
    \FunctionTok{return}\NormalTok{(metriccoorx)}
\NormalTok{  \}}
  
  \CommentTok{\# Run function for each metric}
\NormalTok{  metric4\_coor }\OtherTok{\textless{}{-}} \FunctionTok{metricx\_coor}\NormalTok{(metric4, coorin, res)}
\NormalTok{  metric5\_coor }\OtherTok{\textless{}{-}} \FunctionTok{metricx\_coor}\NormalTok{(metric5, coorin, res)}
\NormalTok{  metric6\_coor }\OtherTok{\textless{}{-}} \FunctionTok{metricx\_coor}\NormalTok{(metric6, coorin, res)}
  
  \CommentTok{\# Rasterize data}
\NormalTok{  r4[[i]]  }\OtherTok{\textless{}{-}} \FunctionTok{rast}\NormalTok{(metric4\_coor[,}\FunctionTok{c}\NormalTok{(}\StringTok{"x"}\NormalTok{, }\StringTok{"y"}\NormalTok{, }\StringTok{"month"}\NormalTok{)], }\AttributeTok{type=}\StringTok{"xyz"}\NormalTok{, }\AttributeTok{crs=}\FunctionTok{crs}\NormalTok{(gridin), }\AttributeTok{extent=}\FunctionTok{ext}\NormalTok{(gridin))}
\NormalTok{  r5[[i]]  }\OtherTok{\textless{}{-}} \FunctionTok{rast}\NormalTok{(metric5\_coor[,}\FunctionTok{c}\NormalTok{(}\StringTok{"x"}\NormalTok{, }\StringTok{"y"}\NormalTok{, }\StringTok{"month"}\NormalTok{)], }\AttributeTok{type=}\StringTok{"xyz"}\NormalTok{, }\AttributeTok{crs=}\FunctionTok{crs}\NormalTok{(gridin), }\AttributeTok{extent=}\FunctionTok{ext}\NormalTok{(gridin))}
\NormalTok{  r6[[i]]  }\OtherTok{\textless{}{-}} \FunctionTok{rast}\NormalTok{(metric6\_coor[,}\FunctionTok{c}\NormalTok{(}\StringTok{"x"}\NormalTok{, }\StringTok{"y"}\NormalTok{, }\StringTok{"month"}\NormalTok{)], }\AttributeTok{type=}\StringTok{"xyz"}\NormalTok{, }\AttributeTok{crs=}\FunctionTok{crs}\NormalTok{(gridin), }\AttributeTok{extent=}\FunctionTok{ext}\NormalTok{(gridin))}

\NormalTok{ \}}

  \CommentTok{\# Assign name of species key to the corresponding raster \textquotesingle{}layer\textquotesingle{}}
  \FunctionTok{names}\NormalTok{(r4) }\OtherTok{\textless{}{-}}\NormalTok{ spskey}
  \FunctionTok{names}\NormalTok{(r5) }\OtherTok{\textless{}{-}}\NormalTok{ spskey}
  \FunctionTok{names}\NormalTok{(r6) }\OtherTok{\textless{}{-}}\NormalTok{ spskey}
  
\NormalTok{  results\_list }\OtherTok{\textless{}{-}} \FunctionTok{list}\NormalTok{(}\AttributeTok{month\_with\_highest\_records =}\NormalTok{ r4, }\AttributeTok{month\_with\_2ndhighest\_records =}\NormalTok{ r5, }\AttributeTok{month\_with\_3rdhighest\_records =}\NormalTok{ r6)}
  \FunctionTok{return}\NormalTok{(results\_list)}

\NormalTok{\}}
\end{Highlighting}
\end{Shaded}

In the following code chunk, we compute metrics 4, 5 and 6.

\begin{Shaded}
\begin{Highlighting}[]
\CommentTok{\# Calculate metrics 4, 5 and 6}
\NormalTok{months\_high\_rec }\OtherTok{\textless{}{-}} \FunctionTok{maxrec\_month123}\NormalTok{(cin, gridin, coorin)}
\end{Highlighting}
\end{Shaded}

\begin{Shaded}
\begin{Highlighting}[]
\CommentTok{\# Calculate metrics 4, 5 and 6}
\NormalTok{month\_1sthigh }\OtherTok{\textless{}{-}}\NormalTok{ months\_high\_rec[[}\DecValTok{1}\NormalTok{]]}
\NormalTok{month\_2ndhigh }\OtherTok{\textless{}{-}}\NormalTok{ months\_high\_rec[[}\DecValTok{2}\NormalTok{]]}
\NormalTok{month\_3rdhigh }\OtherTok{\textless{}{-}}\NormalTok{ months\_high\_rec[[}\DecValTok{3}\NormalTok{]]}
\end{Highlighting}
\end{Shaded}

\begin{Shaded}
\begin{Highlighting}[]
\CommentTok{\# Convert raster layers to data frames}
\NormalTok{spi }\OtherTok{\textless{}{-}} \DecValTok{57}
\NormalTok{df1 }\OtherTok{\textless{}{-}} \FunctionTok{as.data.frame}\NormalTok{(month\_1sthigh[[spi]], }\AttributeTok{xy=}\ConstantTok{TRUE}\NormalTok{)}
\NormalTok{spid }\OtherTok{\textless{}{-}} \FunctionTok{as.numeric}\NormalTok{(}\FunctionTok{colnames}\NormalTok{(df1)[}\DecValTok{3}\NormalTok{])}
\FunctionTok{colnames}\NormalTok{(df1)[}\FunctionTok{colnames}\NormalTok{(df1) }\SpecialCharTok{==} \FunctionTok{colnames}\NormalTok{(df1)[}\DecValTok{3}\NormalTok{]] }\OtherTok{\textless{}{-}} \StringTok{"Value"}
\NormalTok{df1[[}\StringTok{"Value"}\NormalTok{]] }\OtherTok{\textless{}{-}}\NormalTok{ df1[[}\StringTok{"Value"}\NormalTok{]]}

\NormalTok{df2 }\OtherTok{\textless{}{-}} \FunctionTok{as.data.frame}\NormalTok{(month\_2ndhigh[[spi]], }\AttributeTok{xy=}\ConstantTok{TRUE}\NormalTok{)}
\NormalTok{spid }\OtherTok{\textless{}{-}} \FunctionTok{as.numeric}\NormalTok{(}\FunctionTok{colnames}\NormalTok{(df2)[}\DecValTok{3}\NormalTok{])}
\FunctionTok{colnames}\NormalTok{(df2)[}\FunctionTok{colnames}\NormalTok{(df2) }\SpecialCharTok{==} \FunctionTok{colnames}\NormalTok{(df2)[}\DecValTok{3}\NormalTok{]] }\OtherTok{\textless{}{-}} \StringTok{"Value"}
\NormalTok{df2[[}\StringTok{"Value"}\NormalTok{]] }\OtherTok{\textless{}{-}}\NormalTok{ df2[[}\StringTok{"Value"}\NormalTok{]]}

\NormalTok{df3 }\OtherTok{\textless{}{-}} \FunctionTok{as.data.frame}\NormalTok{(month\_3rdhigh[[spi]], }\AttributeTok{xy=}\ConstantTok{TRUE}\NormalTok{)}
\NormalTok{spid }\OtherTok{\textless{}{-}} \FunctionTok{as.numeric}\NormalTok{(}\FunctionTok{colnames}\NormalTok{(df3)[}\DecValTok{3}\NormalTok{])}
\FunctionTok{colnames}\NormalTok{(df3)[}\FunctionTok{colnames}\NormalTok{(df3) }\SpecialCharTok{==} \FunctionTok{colnames}\NormalTok{(df3)[}\DecValTok{3}\NormalTok{]] }\OtherTok{\textless{}{-}} \StringTok{"Value"}
\NormalTok{df3[[}\StringTok{"Value"}\NormalTok{]] }\OtherTok{\textless{}{-}}\NormalTok{ df3[[}\StringTok{"Value"}\NormalTok{]]}


\CommentTok{\# Find the specie scientific name}
\NormalTok{spi\_name }\OtherTok{\textless{}{-}}\NormalTok{ gbif\_tax }\SpecialCharTok{\%\textgreater{}\%}
  \FunctionTok{filter}\NormalTok{(acceptedUsageKey }\SpecialCharTok{\%in\%}\NormalTok{ spid)}
\end{Highlighting}
\end{Shaded}

\begin{Shaded}
\begin{Highlighting}[]
\FunctionTok{ggplot}\NormalTok{() }\SpecialCharTok{+}
  \FunctionTok{geom\_sf}\NormalTok{(}\AttributeTok{data =}\NormalTok{ borders\_eu, }\AttributeTok{fill =} \StringTok{"gray"}\NormalTok{) }\SpecialCharTok{+}
  \FunctionTok{geom\_raster}\NormalTok{(}\AttributeTok{data =}\NormalTok{ df1, }\FunctionTok{aes}\NormalTok{(}\AttributeTok{x =}\NormalTok{ x, }\AttributeTok{y =}\NormalTok{ y, }\AttributeTok{fill =}\NormalTok{ Value)) }\SpecialCharTok{+} 
  \FunctionTok{geom\_sf}\NormalTok{(}\AttributeTok{data =}\NormalTok{ borders\_eu, }\AttributeTok{fill =} \ConstantTok{NA}\NormalTok{, }\AttributeTok{color =} \StringTok{"white"}\NormalTok{) }\SpecialCharTok{+}
  \FunctionTok{scale\_fill\_viridis}\NormalTok{(}\AttributeTok{name=}\StringTok{"Month"}\NormalTok{, }\AttributeTok{option =} \StringTok{"C"}\NormalTok{, }\AttributeTok{direction =} \DecValTok{1}\NormalTok{) }\SpecialCharTok{+} \CommentTok{\# }
  \FunctionTok{theme\_minimal}\NormalTok{() }\SpecialCharTok{+}
  \FunctionTok{labs}\NormalTok{(}
    \AttributeTok{title =} \FunctionTok{bquote}\NormalTok{(}\FunctionTok{atop}\NormalTok{(}\StringTok{"Month with the highest total number of occurrences"}\NormalTok{,}
                        \StringTok{"across all years for "} \SpecialCharTok{*} \FunctionTok{italic}\NormalTok{(.(spi\_name}\SpecialCharTok{$}\NormalTok{scientificName)))),}
    \AttributeTok{x =} \StringTok{"Latitude"}\NormalTok{,}
    \AttributeTok{y =} \StringTok{"Longitude"}\NormalTok{,    }
    \AttributeTok{fill =} \StringTok{"Month"}\NormalTok{)}
\end{Highlighting}
\end{Shaded}

\includegraphics{04_birds_annex1_computing_metrics_3_4_5_files/figure-latex/unnamed-chunk-8-1.pdf}

\begin{Shaded}
\begin{Highlighting}[]
\CommentTok{\# save figure if needed}
\FunctionTok{ggsave}\NormalTok{(}\FunctionTok{here}\NormalTok{(}\StringTok{"output/figures/nb\_birds/ias\_metric4\_id82.png"}\NormalTok{))}
\end{Highlighting}
\end{Shaded}

\begin{Shaded}
\begin{Highlighting}[]
\FunctionTok{ggplot}\NormalTok{() }\SpecialCharTok{+}
  \FunctionTok{geom\_sf}\NormalTok{(}\AttributeTok{data =}\NormalTok{ borders\_eu, }\AttributeTok{fill =} \StringTok{"gray"}\NormalTok{) }\SpecialCharTok{+}
  \FunctionTok{geom\_raster}\NormalTok{(}\AttributeTok{data =}\NormalTok{ df2, }\FunctionTok{aes}\NormalTok{(}\AttributeTok{x =}\NormalTok{ x, }\AttributeTok{y =}\NormalTok{ y, }\AttributeTok{fill =}\NormalTok{ Value)) }\SpecialCharTok{+} 
  \FunctionTok{geom\_sf}\NormalTok{(}\AttributeTok{data =}\NormalTok{ borders\_eu, }\AttributeTok{fill =} \ConstantTok{NA}\NormalTok{, }\AttributeTok{color =} \StringTok{"white"}\NormalTok{) }\SpecialCharTok{+}
  \FunctionTok{scale\_fill\_viridis}\NormalTok{(}\AttributeTok{name=}\StringTok{"Month"}\NormalTok{, }\AttributeTok{option =} \StringTok{"C"}\NormalTok{, }\AttributeTok{direction =} \DecValTok{1}\NormalTok{) }\SpecialCharTok{+} \CommentTok{\# }
  \FunctionTok{theme\_minimal}\NormalTok{() }\SpecialCharTok{+}
  \FunctionTok{labs}\NormalTok{(}
    \AttributeTok{title =} \FunctionTok{bquote}\NormalTok{(}\FunctionTok{atop}\NormalTok{(}\StringTok{"Month with the second highest total number of occurrences"}\NormalTok{,}
                        \StringTok{"across all years for "} \SpecialCharTok{*} \FunctionTok{italic}\NormalTok{(.(spi\_name}\SpecialCharTok{$}\NormalTok{scientificName)))),}
                          
    \AttributeTok{x =} \StringTok{"Latitude"}\NormalTok{,}
    \AttributeTok{y =} \StringTok{"Longitude"}\NormalTok{,    }
    \AttributeTok{fill =} \StringTok{"Month"}\NormalTok{)}
\end{Highlighting}
\end{Shaded}

\includegraphics{04_birds_annex1_computing_metrics_3_4_5_files/figure-latex/unnamed-chunk-10-1.pdf}

\begin{Shaded}
\begin{Highlighting}[]
\CommentTok{\# save figure if needed}
\FunctionTok{ggsave}\NormalTok{(}\FunctionTok{here}\NormalTok{(}\StringTok{"output/figures/nb\_birds/ias\_metric5\_id82.png"}\NormalTok{))}
\end{Highlighting}
\end{Shaded}

\begin{Shaded}
\begin{Highlighting}[]
\FunctionTok{ggplot}\NormalTok{() }\SpecialCharTok{+}
  \FunctionTok{geom\_sf}\NormalTok{(}\AttributeTok{data =}\NormalTok{ borders\_eu, }\AttributeTok{fill =} \StringTok{"gray"}\NormalTok{) }\SpecialCharTok{+}
  \FunctionTok{geom\_raster}\NormalTok{(}\AttributeTok{data =}\NormalTok{ df3, }\FunctionTok{aes}\NormalTok{(}\AttributeTok{x =}\NormalTok{ x, }\AttributeTok{y =}\NormalTok{ y, }\AttributeTok{fill =}\NormalTok{ Value)) }\SpecialCharTok{+} 
  \FunctionTok{geom\_sf}\NormalTok{(}\AttributeTok{data =}\NormalTok{ borders\_eu, }\AttributeTok{fill =} \ConstantTok{NA}\NormalTok{, }\AttributeTok{color =} \StringTok{"white"}\NormalTok{) }\SpecialCharTok{+}
  \FunctionTok{scale\_fill\_viridis}\NormalTok{(}\AttributeTok{name=}\StringTok{"Month"}\NormalTok{, }\AttributeTok{option =} \StringTok{"C"}\NormalTok{, }\AttributeTok{direction =} \DecValTok{1}\NormalTok{) }\SpecialCharTok{+} \CommentTok{\# }
  \FunctionTok{theme\_minimal}\NormalTok{() }\SpecialCharTok{+}
  \FunctionTok{labs}\NormalTok{(}
    \AttributeTok{title =} \FunctionTok{bquote}\NormalTok{(}\FunctionTok{atop}\NormalTok{(}\StringTok{"Month with the third highest total number of occurrences"}\NormalTok{,}
                        \StringTok{"across all years for "} \SpecialCharTok{*} \FunctionTok{italic}\NormalTok{(.(spi\_name}\SpecialCharTok{$}\NormalTok{scientificName)))),}
    \AttributeTok{x =} \StringTok{"Latitude"}\NormalTok{,}
    \AttributeTok{y =} \StringTok{"Longitude"}\NormalTok{,    }
    \AttributeTok{fill =} \StringTok{"Month"}\NormalTok{)}
\end{Highlighting}
\end{Shaded}

\includegraphics{04_birds_annex1_computing_metrics_3_4_5_files/figure-latex/unnamed-chunk-12-1.pdf}

\begin{Shaded}
\begin{Highlighting}[]
\CommentTok{\# save figure if needed}
\FunctionTok{ggsave}\NormalTok{(}\FunctionTok{here}\NormalTok{(}\StringTok{"output/figures/nb\_birds/ias\_metric6\_id82.png"}\NormalTok{))}
\end{Highlighting}
\end{Shaded}

\subsubsection{Saving data sets as individual
tiffs}\label{saving-data-sets-as-individual-tiffs}

\begin{Shaded}
\begin{Highlighting}[]
\CommentTok{\# Save files}
\FunctionTok{writeRaster}\NormalTok{(month\_1sthigh, }\FunctionTok{here}\NormalTok{(}\StringTok{"output/datacubes/tif\_metrics/birds/04\_birds\_month\_1sthigh.tif"}\NormalTok{), }\AttributeTok{datatype =} \StringTok{\textquotesingle{}INT4U\textquotesingle{}}\NormalTok{, }\AttributeTok{overwrite =} \ConstantTok{TRUE}\NormalTok{)}
\FunctionTok{writeRaster}\NormalTok{(month\_2ndhigh, }\FunctionTok{here}\NormalTok{(}\StringTok{"output/datacubes/tif\_metrics/birds/05\_birds\_month\_2ndhigh.tif"}\NormalTok{), }\AttributeTok{datatype =} \StringTok{\textquotesingle{}INT4U\textquotesingle{}}\NormalTok{, }\AttributeTok{overwrite =} \ConstantTok{TRUE}\NormalTok{)}
\FunctionTok{writeRaster}\NormalTok{(month\_3rdhigh, }\FunctionTok{here}\NormalTok{(}\StringTok{"output/datacubes/tif\_metrics/birds/06\_birds\_month\_3rdhigh.tif"}\NormalTok{), }\AttributeTok{datatype =} \StringTok{\textquotesingle{}INT4U\textquotesingle{}}\NormalTok{, }\AttributeTok{overwrite =} \ConstantTok{TRUE}\NormalTok{)}
\end{Highlighting}
\end{Shaded}


\end{document}
